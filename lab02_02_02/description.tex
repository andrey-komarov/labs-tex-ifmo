\section{Содержание работы}

\sub

\subsection{Уравнение процесса. Характеристики затухания}
Простой колебательный контур состоит из последовательных индуктивности
L, ёмкости C, и активного сопротивления R. Если предварительно запасти 
энергию, например, зарядив конденсатор от внешнего источника тока, а 
затем подключить конденсатор к катушке индуктивности, то в образовавшемся
изолированном контуре возникнут \emph{свободные} электромагнитные колебания.

Действительно, при разряде конденсатора появляются изменяющиеся во 
времени ток и пропорциональное ему магнитное поле. Меняющееся магнитное
поле порождает в контуре ЭДС самоиндукции \WTF , которая по закону Ленца
сначала замедляет скорость разряда конденсатора, а после того, как 
конденсатор полностью разрядится, продолжает поддерживать ток в прежнем 
направлении. В результате происходит перезарядка конденсатора. Затем
процесс разряда конденсатора продолжается, но в обратном направлении и т.д.
Возникающие свободные колебания заряда $q$, тока $I$ и напряжений $U$ на
элементах контура совершаются с циклической частотой $\omega$ (периодом
$T = 2\pi/\omega$), а колебания электрической $W_e = \frac{CU^2}2$ и 
магнитной $W_m = \frac{LI^2}2$ энергией $I$ с удвоенной частотой $2\omega$
(максимумы энергий появляются дважды за период $T$).

Вследствие джоулевых потерь в активном сопротивлении контура $R$ часть 
энергии колебаний превращается в теплоту, что приводит к затуханию колебаний.
При больших величинах $R$ колебания могут вообще не возникнуть~--- 
наблюдается апериодический разряд конденсатора.

Найдём уравнение, описывающее свободные затухающие колебания в контуре.
Заряд $q$ на конденсаторе, напряжении на нём $U$, ток в контуре $I$ и 
ЭДС самоиндукции $\varepsilon_s$ связаны соотношениями 
\begin{equation}
\label{1}
q = CU, I = \frac{dq}{dt} = C\cdot\frac{dU}{dt}, 
\varepsilon_s = -L \cdot \frac{dI}{dt} = -LC\cdot\frac{d^2U}{dt^2}
\end{equation}

По закону Кирхгофа для полной цепи имеем
\begin{equation}
\label{2}
IR = -U + \varepsilon_s
\end{equation}

С учётом соотношений (\ref{1}) уравнение (\ref{2}) для переменной $U$ 
приобретает вид 
\begin{equation}
\label{3}
\frac{d^2U}{dt^2} + 2\delta\frac{dU}{dt} + \omega_0^2U = 0, 
\end{equation}
где введены обозначения $\omega_0 = \frac1{\sqrt{LC}}$~--- собственная 
частота контура. Легко показать, что точно такой же вид имеют уравнения 
для заряда конденсатора $q$ и тока $I$.

Из теории известно, что полученное дифференциальное уравнение второго 
порядка с постоянными коэффициентами в зависимости от соотношения между 
$\delta$ и $\omega_0$ имеет решение~--- функции, по разному меняющиеся 
по времени.

При условии $\delta < \omega_0$ (малое затухание) уравнение (\ref{3}) 
имеет решение в виде
\begin{equation}
\label{4}
U = U_0 \cdot e^{-\delta t} \cdot \cos \omega t, где \omega = 
\sqrt{\omega_0^2 - \delta^2}, 
\end{equation}
которое описывает затухающий колебательный процесс.

Затухание нарушает периодичность колебаний и строгое применение понятия
периода и частоты к ним не применимо. Однако при малом затухании условно
пользуются понятием периода как промежутка времени между последующими
максимумами (или минимумами) колеблющейся величины. С учётом этой оговорки 
период свободных затухающих колебаний в контуре равен
\begin{equation}
\labal{5}
T = \frac{2\pi}\omega = \frac{2\pi}{
\sqrt{\frac1{LC} - \frac{R^2}{4L^2}}
}
\end{equation}

С увеличением затухания период колебаний растёт, обращаясь в бесконечность 
при $\delta = \omega_0$, т.е. движение перестаёт быть периодическим. В
данном случае ($\delta \geq \omega_0$) напряжение на конденсаторе 
приближается к нулю при $t \to 0$ и уже будет описываться функцией, 
отличной от вида (\ref{4})

Такой процесс называется \emph{апериодическим}. Переход к нему происходит
при величине сопротивления контура $R \geq R_{кр} = 2\sqrt{\frac{L}C}$.

В качестве меры затухания колебательного процесса кроме коэффициента 
затухания $\delta$ используются и другие характеристики:
\begin{enumerate}
\item \emph{время релаксации} $\tau$~--- интервал времени, за который 
	амплитуда колебаний уменьшается в $e = 2{,}72$ раза:
	\begin{equation}
		\label{6}
		\tau = \frac1\delta = \frac{2L}R.
	\end{equation}

\item \emph{логарифмический декремент затухания} $\lambda$~--- величина,
	определяемая как натуральный логарифм отношения двух амплитуд $U(t)$ и 
	$U(t + T)$, разделённых интервалом времени, равным периоду колебаний $T$:
	\begin{equation}
		\label{7}
		\lambda = \ln \frac{U(t)}{U(t + T)} = \delta T = \frac{R}{2L} \cdot T.
	\end{equation}
	
	На практике измеряется отношение амплитуд $U(t)$ и $U(t + nT)$, отстоящих 
	друг от друга на $n$ периодов, тогда
	\begin{equation}
		\label{7a}
		\lambda = \frac1n\ln\frac{U(t)}{U(t + nT)}.
	\end{equation}
	
	Из формулы (\ref{7a}) вытекает смысл $\lambda$ как величины, обратной
	числу периодов, за время которых амплитуда колебаний уменьшается в 
	$e = 2{,}71$ раза.

\item \emph{добротность контура} $Q$~--- величина, определяемая соотношением
	\begin{equation}
		\label{8}
		Q = 2\pi\frac{W(t)}{W(t) - W(t + T)},
	\end{equation}
	где $W$~--- запасённая энергия, $\Delta W = W(t) - W(t + T)$~--- 
	средняя потеря энергии за период $T$.
	
	При малых затуханиях ($\delta^2 \leq \omega_0^2$) величина добротности 
	равна
	\begin{equation}
		\label{9}
		Q = \frac\pi\lambda = \pi N_\varepsilon,
	\end{equation}
	где $N_\varepsilon$~--- число колебаний, происходящих за время 
	релаксации $\tau$. Если выразить добротность через параметры контура,
	то получим 
	\begin{equation}
		\label{10}
		Q = \frac1R\cdot\sqrt{\frac{L}C}.
	\end{equation}
\end{enumerate}



\subsection{Фазовая плоскость}
В ряде случаев удобно изучать колебательные и нелинейные процессы в 
системе координат $(I, U)$~--- <<ток-напряжение>>. В механике
аналогичными координатами являются скорость и перемещение. Плоскость
таких координат носит название \emph{плоскости состояний} или
\emph{фазовой плоскости}, а кривая, изображающая зависимость этих 
координат называется \emph{фазовой кривой}.

Рассмотрим фазовую кривую для процессов в $LCR$-контуре. Для нахождения
силы тока продифференцируем функцию $U(t)$ (\ref{4}) по времени:
$$
I = C\frac{dU}{dt} = CU_0e^{-\delta t} \cdot (-\delta \cos\omega t + \omega \sin\omega t).
$$

Умножим и разделим это выражение на $\omega_0 = \sqrt{\omega^2 + \delta^2}$.

\begin{equation}
\label{11}
\dot I = I_0 \cdot e^{-\delta t} \cos(\omega t + \psi).
\end{equation}

Так как $\cos \psi < 0$, а $\sin\psi > 0$, то фазовый сдвиг между током и 
напряжением заключен в интервале $\frac\pi2 < \psi < \pi$. Это означает, 
что ток в контуре опережает по фазе напряжение на конденсаторе не более, 
чем на $\frac\pi2$.

Фазовая кривая $I(U)$ описывается в параметрической форме системой из двух 
уравнений
\begin{equation}
	\label{12}
	\begin{cases}
		U & = & U_0 e^{-\delta t} \cos \omega t \\
		I & = & I_0 e^{-\delta t} \cos (\omega t + \psi) \\
	\end{cases}
\begin{equation}

При $R = 0$ ($\delta = 0$) опережение тока по фазе составляет $\psi = \frac\pi2$
и фазовая кривая будет представлять собой эллипс, как в случае сложения
двух взаимно перпендикулярных колебаний с постоянными амплитудами, сдвинутых
по фазе на четверть периода. В реальной ситуации при наличии затухания 
($R > 0$) амплитуды напряжения и тока в контуре непрерывно убывают, не
повторяясь через период $T$, и фазовая кривая получается незамкнутой.

