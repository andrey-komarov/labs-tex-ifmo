\section{Порядок выполнения работы}

\begin{enumerate}
    \item С помощью лабораторного барометра определить текущее 
    атмосферное давление $P_0$ и записать его значение в протокол работы.

    \item Включить цифровой измерительный прибор в режиме измерения 
    температуры и давления (включение производит дежурный инженер). 
    Установить рабочий объем цилиндра $100$ мл 
    (обратите внимание на положение визирной стрелки на маховике штока 
    поршня). Залить в термостат четыре с половиной кружки воды комнатной 
    температуры. Аккуратно поместить в термостат цилиндр $1$ с датчиком 
    температуры так, чтобы опорная площадка цилиндра легла на горлышко 
    термостата. 
    
    \item Подождать пока показания давления престанут изменяться 
    (приблизительно $5$--$10$ минут). При этом рабочий объем газ придет 
    в тепловое равновесие с термостатом. Записать значение температуры 
    термостата (около $20^{\circ}$) в таблицу $1.1$. Последовательно 
    изменяя рабочий объём газа, начиная со $100$ мл, с шагом $10$ мл
    сначала в сторону уменьшения до $50$ мл, затем в сторону увеличения
    до $140$ мл и обратно до $50$ мл, затем снова до $100$ мл дважды
    измерить разность давлений $\Delta P$ для каждого значения $V_ц$ в 
    таблице 1.1. Результаты заносятся в ячейки третьего и четвертого 
    столбцов таблицы по часовой стрелке. Изменение объема можно 
    контролировать, не вынимая цилиндр из термостата "--- уменьшению
    на $10$ мл соответствует два оборота маховика по часовой стрелке.

    \item Аккуратно вынуть рабочий цилиндр с датчиком температуры из 
    термостата, положить на поддон. Отлить из термостата приблизительно 
    три четверти кружки воды. Вылить воду из кружки в емкость для 
    использованной воды. Налить из чайника три четверти кружки горячей
    воды в термостат. Перемешать воду в термосе. Поместить в термостат 
    цилиндр с датчиком температуры. В термостате должна установиться 
    новая температура  (около $30^{\circ}C$). Повторить измерения п.3. 
    Занести результаты в таблицу 1.2 , аналогичную таблице 1.1.

    \item Последовательно изменяя температуру термостата до значений 
    $t_3(40^{\circ}C)$, $t_4(50^{\circ}C)$ и $t_5(60^{\circ}C)$, 
    как описано в пункте 4, записать получившиеся значения температур 
    и произвести измерения п.3. Занести результаты в таблицы 
    1.3, 1.4, 1.5, аналогичные таблице 1.1.
    
    \item После выполнения всех измерений выключить цифровой 
    измерительный прибор, вынуть цилиндр с датчиком температуры и 
    положить на поддон. Вылить воду из термостата в емкость для 
    использованной воды.
    
\end{enumerate}

