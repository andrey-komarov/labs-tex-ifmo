\section{Теоретические основы лабораторной работы}

В~том случае, когда состояние газа далеко от области фазовых превращений, 
его с достаточной степенью точности можно считать идеальным. 
В~качестве идеального газа в работе используется обычный атмосферный воздух. 

Для произвольной массы $m$ идеального газа справедливо следующее уравнение состояния:
\begin{equation}
    \label{sost}
    PV = \frac{m}{\mu} RT
\end{equation}
где $P$ "--- давление, $V$ "--- объем, $\mu$ "--- молярная масса, 
$T$ "--- абсолютная температура газа, $R$ "--- универсальная газовая 
постоянная. Это уравнение называется уравнением Менделеева-Клапейрона.

Нулю абсолютной температуры по шкале Цельсия соответствует значение 
$t_0 = -273.15^{\circ}$. Градусы шкалы абсолютной температуры~(шкалы Кельвина) 
и шкалы Цельсия выбраны одинаковыми. Поэтому значение абсолютной 
температуры связано со значением температуры по шкале Цельсия формулой 
\begin{equation}
    \label{temp}
    T = t({}^{\circ}C) - t_0 = t({}^{\circ}C) + 273.15^{\circ}C
\end{equation}

Пусть исследуемый газ находиться в цилиндре с контролируемым рабочим 
объемом $V_ц$ (см. рис. 1), масса газа в цилиндре $m_ц$.
Температура $t$ цилиндра с газом поддерживается постоянной. 

Датчик давления, работающий при комнатной температуре, вынесен за пределы 
рабочего объема и соединен с последним трубкой. 
Объем газа $V_х$ в этой трубке мал по сравнению с рабочим объемом $V_ц$. 
В соединительной трубке также находится газ массой $m_x$ при некоторой 
неизвестной средней температуре $t_x$, лежащей в интервале от комнатной 
температуры до температуры $t$ рабочего объема.

В работе измеряется зависимость давления $P$ газа от величины рабочего 
объема $V_{\mbox{ц}}$ при разных значениях температуры $t$ (от $20^{\circ}С$ 
до $60^{\circ}С$). Выведем соотношение, связывающее рабочий объем и 
давление газа при постоянной температуре. 
Общее количество вещества в рабочем объеме и соединительной трубке
\begin{equation}
    \label{vol}
    v = \frac{m_{\mbox{ц}} + m_x}{\mu}
\end{equation}
в течение всей работы остается постоянным. 
Выражая массы газа $m_{\mbox{ц}}$ и $m_x$ из уравнения состояния (\ref{sost}), 
абсолютную температуру из соотношения (\ref{temp}), и подставляя 
найденные выражения в формулу (\ref{vol}), получим
\begin{equation}
    \nu = \frac{PV_{\mbox{ц}}}{R(t - t_0)} + \frac{PV_x}{R(t_x - t_0)}
\end{equation}
Из этого уравнения найдем искомое соотношение:
\begin{equation}
    \label{cyl_vol}
    V_{\mbox{ц}} = \frac{\nu R(t - t_0)}{P} + \frac{V_x(t - t_0)}{t_x - t_0}
\end{equation}

Из-за перераспределения газа между объемами $V_{\mbox{ц}}$ и $V_х$ в 
процессе измерения температура $t_х$ может изменяться. 
Однако, при относительно малой величине $V_х$ изменением второго 
слагаемого в формуле (\ref{cyl_vol}) можно пренебречь. Поэтому при 
неизменной температуре $t$ зависимость рабочего объема $V_{\mbox{ц}}$ от 
обратного давления $\frac1P$ является линейной. Угловой коэффициент 
этой зависимости 
\begin{equation}
    \label{angle}
    K = \nu R (t - t_0)
\end{equation}
в свою очередь, линейно меняется с температурой и обращается в нуль при 
абсолютном нуле температур. Таким образом, изучение зависимости $K(t)$ 
позволяет найти значение $t_0$.

Рассмотрим другой, более точный, способ определения величины $t_0$. 
Если для разных температур измерение давления проводить при одних и тех 
же значениях объема, то полученные данные легко преобразуются в 
зависимость давления от температуры при разных значениях рабочего 
объема газа. Теоретический вид этой зависимости получается из 
уравнения (\ref{cyl_vol}):
\begin{equation}
    \label{P_from_VT}
    P = \frac{\nu R(t - t_0)}{V_ц(1 + x(t))} \approx
    \frac{\nu R(t - t_0)}{V_ц}(1 - x(t))
\end{equation}
где $x(t) = \frac{V_x(t - t_0)}{V_ц(t_x - t_0)}$. 
Справедливость приближенного равенства в формуле (\ref{P_from_VT}) 
обусловлена тем, что значения функции $x(t)$ малы, и для малых $x$ можно 
воспользоваться формулой приближенных вычислений:
\begin{equation}
    (1 + x)^{\alpha} \approx 1 + \alpha x
\end{equation}
В данном случае $\alpha = -1$.

При неизменном рабочем объеме $V_ц$ график зависимости давления от 
температуры в соответствии с формулой (\ref{P_from_VT}) должен быть почти 
линейным. Причем давление должно обращаться в нуль как раз при $t = t_0$. 
Из-за малости функции отклонение от линейности невелико, и при измерении 
в ограниченном диапазоне температур практически незаметно. Но, если 
искать значение $t_0$ с помощью линейной аппроксимации экспериментальной 
зависимости $P(t)$, продолжая (экстраполируя) аппроксимирующую прямую 
до пересечения с осью $t$, то найденное приближенное значение 
$\tilde{t}_0$ окажется систематически смещенным влево 
относительно истинного значения $t_0$(см. рис. 2). Причина этого в 
следующем. Величина $x(t)$ в первом приближении линейно растущая функция 
температуры, с учетом этого график функции $P(t)$ из уравнения (\ref{P_from_VT}) 
оказывается параболой, выпуклой вверх. Аппроксимирующая прямая, 
параметры которой найдены по точкам в рабочем диапазоне температур, идет 
практически по касательной к этому графику, <<промахиваясь>> мимо 
истинного значения $t_0$, как изображено на рис. 1. Однако, можно
показать, что разность $\tilde{t}_0 - t_0$ при малом отношении 
$\frac{V_x}{V_ц}$ должна убывать обратно пропорционально объему $V_ц$. 
Поэтому, правильное значение температуры абсолютного нуля может быть 
найдено как предел:
\begin{equation}
    \label{lim}
    t_0 = \lim\limits_{\frac1{V_ц} \to 0} \tilde{t}_0
\end{equation}
линейным продолжением графика зависимости $\tilde{t}_0$ от $\frac1{V_ц}$ 
к значению $\frac1{V_ц} \to 0$.
