\section{Теоретические основы лабораторной работы}

Взаимодействие между неподвижными электрически заряженными телами осуществляется
посредством электрического поля. При этом каждое заряженное тело создает в окружающем
пространстве поле, воздействующее на другие заряженные тела, и само это тело испытывает на
себе воздействие электрических полей, созданных окружающими телами. Если
заряды-источники неподвижны, то их электрическое поле стационарно, т.е. не
изменяется с течением времени. Такое поле называют электростатическим. Силовой характеристикой электрического
поля служит вектор его напряженности. Этот вектор в данной точке пространства определяется
соотношением
\begin{equation}
    \label{electric}
    \overline{E} = \frac{\overline{F}}{q}
\end{equation}

где $\overline{F}$ -– сила, действующая на неподвижный заряд $q$,
помещенный в данную точку. Заряд $q$ в формуле (\ref{electric}), с помощью
которого детектируется электрическое поле, называется «пробным». Для графического изображения электростатических полей используют силовые линии.
Силовыми линиями (линиями напряженности) называют линии, касательные к которым в
каждой точке совпадают с направлением вектора напряженности в этой точке. Силовые линии
электростатического поля разомкнуты. Они начинаются на положительных зарядах и
оканчиваются на отрицательных зарядах (в частности, они могут уходить в бесконечность или
приходить из бесконечности).

Энергетической характеристикой электрического поля является его потенциал. Потенциалом
в данной точке поля называется скалярная величина
\begin{equation}
    \label{potential}
    \varphi = \frac{W_п}{q}
\end{equation}
где $W_п$ – потенциальная энергия заряда $q$, помещенного в данную точку. При
перемещении заряда $q$ из точки с потенциалом $\varphi_1$ в точку с
потенциалом $\varphi_2$ силы электростатического поля совершают над зарядом
работу 
\begin{equation}
    \label{work}
    A = q(\varphi_1 - \varphi_2)
\end{equation}
Геометрическое место точек, в которых потенциал имеет одинаковую величину, называется
эквипотенциальной поверхностью.

Напряженность и потенциал электростатического поля связаны друг с другом
соотношениями
\begin{equation}
    \label{gradient}
   \overline{E} = -grad\varphi
\end{equation}
\begin{equation}
    \label{integral}
    \varphi_1 - \varphi_2 = \int\limits_1^2(\overline{E},d\overline{l})
\end{equation}
Вектор градиента (градиент) потенциала в формуле (\ref{gradient}) определяется через
частные производные потенциала по декартовым координатам $x, y, z$ :

\begin{equation}
    \label{gradient2}
   grad\varphi = \overline{e_x}\frac{\partial \varphi}{\partial x} +
   \overline{e_y}\frac{\partial \varphi}{\partial y} +
   \overline{e_z}\frac{\partial \varphi}{\partial z}
\end{equation}
Здесь $\overline{e_x}$ , $\overline{e_y}$ ,$\overline{e_z}$ – единичные вектора положительных
направлений (орты) координатных осей $Ox$, $Oy$, $Oz$. Направление градиента
потенциала в данной точке совпадает с направлением быстрейшего возрастания потенциала, а его величина равна быстроте изменения потенциала на
единицу перемещения в этом направлении. Направление вектора $\overline{E}$
напряженности электростатического поля в соответствии с формулой (\ref{gradient}) противоположно направлению
градиента. Следовательно, вектор напряженности направлен в сторону
наибыстрейшего убывания потенциала. Кроме того, из формулы (\ref{integral}) следует,
что вектор $\overline{E}$ перпендикулярен к эквипотенциальной поверхности в любой ее точке.

Если известны потенциалы  $\varphi_1$ и $\varphi_2$ двух точек, лежащих на одной
силовой линии, то средняя напряженность между этими точками вычисляется по формуле
\begin{equation}
    \label{electric2}
   E_{12} = \frac{\varphi_1 - \varphi_2}{l_{12}} 
\end{equation}
где $l_{12}$ –- длина участка силовой линии между точками. Если относительное
изменение локального значения напряженности между выбранными точками невелико, то формула
(\ref{electric2}) дает значение близкое к напряженности на середине участка 1-2.

В лабораторной работе исследуется пространственное распределение потенциала и напряженности электростатического
поля для двух плоских моделей, в одной из которых электростатическое поле
совпадает с полем плоского конденсатора , в другой –- с полем цилиндрического
конденсатора. Внутри плоского конденсатора вдали от краев пластин электрическое поле однородно
$\overline{E} = const$, и потенциал равномерно возрастает при
движении вдоль координатной оси $x$ от отрицательной обкладки к положительной по формуле
\begin{equation}
    \label{potencial3}
   \varphi(x) = \varphi_0 + Ex 
\end{equation}
где  $\varphi_0$ –- потенциал отрицательной пластины, $E$ –- модуль вектора
электрической напряженности. Внутри цилиндрического конденсатора модуль
электрической напряженности спадает обратно
пропорционально расстоянию $r$ от оси ( $E \sim 1/r$), и, если
внутренняя обкладка заряжена отрицательно, потенциал изменяется в соответствии с
формулой
\begin{equation}
    \label{potencial34}
   \varphi(r) = \varphi_0 + \frac{U\ln(r/r_0)}{\ln(r_1/r_0)} 
\end{equation}
где $\varphi_0$ –- потенциал внутренней обкладки; $U$ –- разность
потенциалов между обкладками; $r_0, r_1$ –- радиусы внутренней и внешней
обкладок соответственно.

